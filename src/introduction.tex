\chapter*{Introduction}
\addcontentsline{toc}{chapter}{Introduction}

Writing a compelling thesis introduction is essential to set the stage for your research. The following steps provide a structured approach to crafting an introduction similar in style and depth to advanced scientific theses.

\begin{enumerate}
    \item \textbf{Start with a Broad Context:}  
    Introduce the global relevance of your topic using statistics, authoritative reports, and recent studies. This helps establish the importance of your research.

    \item \textbf{Narrow Down to Specific Issues:}  
    Identify the specific subset of the broader topic that your thesis addresses. Highlight key challenges and affected populations.

    \item \textbf{Describe the Scientific or Clinical Problem:}  
    Explain the mechanisms or consequences of the issue. Introduce key concepts that will be central to your thesis.

    \item \textbf{Review Existing Solutions and Their Limitations:}  
    Summarize current approaches and highlight their shortcomings. This sets the stage for your contribution.

    \item \textbf{Introduce Your Research Focus:}  
    Clearly state the main objective of your thesis. Use phrases like ``the focus of this thesis is...'' or ``this work aims to...''

    \item \textbf{List Specific Aims or Objectives:}  
    Use a numbered list to break down your goals. This helps readers quickly grasp the scope and direction of your work.

    \item \textbf{Outline the Thesis Structure:}  
    Provide a brief overview of each chapter to orient the reader and show the logical flow of your work.

    \item \textbf{Mention Collaborations or Context:}  
    If applicable, include details about collaborative efforts, research stays, or institutional partnerships that shaped your work.
\end{enumerate}

This structure ensures clarity, relevance, and coherence, helping readers understand the motivation and scope of your research from the outset.