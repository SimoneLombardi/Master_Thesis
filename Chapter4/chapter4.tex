%%%%%%%%%%%%%%%%%%%%%%%%%%%%%%%%%%%%%%%%%%%%%%%%%%%%%%%%%%%%%%%%%%%%%%%%%%%%%%%%
%2345678901234567890123456789012345678901234567890123456789012345678901234567890
%        1         2         3         4         5         6         7         8
% THESIS CHAPTER

\chapter{Architecture implementation}
\label{chap:fourth}
\graphicspath{{Chapter4/Figures/PNG/}{Chapter4/Figures/JPG/}{Chapter4/Figures/JPEG/}{Chapter4/Figures/}}


\section{System Description}
% descrivere come si presenta il sistema, le scelte fatte e le 
% challenge relative alla mia tesi
%To implement a unified architecture i started from a series of specification discussed also with my supervisors, and taking inspiration from the previous work of the team.
%Since the data structure for the \textbf{Kuka} and for the \textbf{UR10e}, comprised of the necessary part for the simulation the idea was to incorporate those part in my work and expand the existing methods to create an unified architecture.

The robotic system I worked with was composed of two articulated industrial robot, namely a \textbf{Kuka KR150} from \textit{Kuka} and a \textbf{UR10e} form \textit{Universal Robot}.

\begin{figure}[H]
	\centering
	\includegraphics[scale=0.2]{Real_system.jpg}
	\caption{photo of the real system, inside the workcell}
	\label{fig:RealSystem}
\end{figure}
\noindent The \textit{UR10e} was mounted on the flange of the \textit{Kuka KR150}, and It holds the sensory apparatus of the system, mounted via purposfully 3d printed sleeves that can incorporate the pressure and proximity sensor in a small footprint and easly removable manner.
\begin{figure}[H]
	\centering
	\begin{subfigure}{0.65\textwidth}
		\centering
		\includegraphics[width=\linewidth]{sensor_sleeve.png}
		\caption{Ur10e sensorized sleeve}
	\end{subfigure}
	\hfill
	\begin{subfigure}{0.3\textwidth}
		\centering
		\includegraphics[width=\linewidth]{tof.png}
		\caption{Time of Flight sensor}
		\vspace{0.5em} % vertical spacing
		\includegraphics[width=\linewidth]{pressure_sensor.png}
		\caption{Pressure sensor board}
	\end{subfigure}
	
	\label{fig:sensorized_array}
\end{figure}
\noindent As shown in the picture the \textit{time of flight sensor} ar contained in strips that wrap around the links body, this allows the robot to have $360°$ around each sensorized link. The pressure sensor instead are placed in the black section in between each \textit{ToF} strip. \\
\noindent In the photo it is possible to see a single sensor board, each array of sensor has: forty-eight boards and each sleeve has two array of sensor.  

\section{Cinematic simulation}
As stated in \ref{chap:first} I used a cinematic simulation to test my work. The main component are described in the following image:
\begin{figure}[H]
	\centering
	\includegraphics[scale=0.3]{simulation_node_diagram.png}
	\caption{Cinematic simulation diagram}
	\label{fig:simulation_nodes_diag}
\end{figure}
In the images most of the nodes related to visualization of the robot motion are omitted, but the core functionalities are present.
\begin{itemize}
	\item \textbf{State broadcasting}: the nodes in blue are responsible for the publication of the \textit{joint states} of the two robots.
	\item \textbf{Environment handling}: the nodes in yellow handle the publication of the environment pointcloud, and the computation of the minimum distances for the robot links.
	\item \textbf{Control}: the nodes in red contains the control loop, and are responsible for publishing the forward kinematics. 
\end{itemize}
\noindent In the end all is visualized by Rviz.
For the \textit{Ur10e} it is available a docker cinematic simulator, \textit{ursim polyscope} developed directly by by \textit{Universal robots}, and for the \textit{Kuka} a simulator was developed starting from the \textit{urdf} representation of the robot and the readly available \textit{ros2} pkgs such as \textit{robot\_state\_broadcaster}.\\
\noindent The \textit{control} nodes contains the logic of the action server I developed for this thesis.

\section{Unified Robot Architecture: JointRobotTP Class}
In this section Im going to explain the implementation of the \textit{unified robot architecture} that is mainly contained in the previously mentioned node: \textit{joint\_robot\_tp\_node}.\\
\noindent All the code was developed in the \codetxt{darkgray}{c++} language, for performance and the matrix computation is done using the library \codetxt{darkgray}{Eigen}.\\
\noindent Here is a simplified diagram of the main class developed to handle the control of the robotic system:\\
\begin{figure}[H]
	\centering
	\includegraphics[scale=0.5]{simplified_class_structure.png}
	\caption{Simplified structure of the class JointRobotTP}
	\label{fig:JointRobotTP simplified}
\end{figure}
\noindent As shown in \ref{fig:simulation_nodes_diag} there are three control nodes, the two nodes for the robots are members of the  \codetxt{darkgray}{JointRobotTP} class.
They were developed during the \textit{Sesto senso project} and contain the data structure for the following application:
\begin{itemize}
	\item \textit{Transformation} listening and broadcasting, using the \textit{tf2\_ros} pkg.
	\item \textit{Inverse kinematics} using using the \textit{KDL} pkg to recursively compute the jacobians of the robots. Starting from the \codetxt{darkgray}{urdf} description of the robots.
\end{itemize}
\noindent
The transformation buffer contain the kinematic chain of the two robots, as \codetxt{darkgray}{geometry\_msgs::msg::TransformStamped}. These transformation are periodically updated trough a topic, published by the \textit{robot state publisher}.
The \textit{tf2\_ros} also allows to retrive specific frame to frame trasformation, using the frames id, defined in the \codetxt{darkgray}{urdf} files.\\

\noindent The main goals I wanted to pursue during development were the following:

\begin{enumerate}
	\item Have an efficient way to send commands to the unified robot.
	\item Use a data structure that allows for easy addition and removal of tasks from the hierarchy.
	\item Keep a degree of separation between the robot information and the control algorithm.
	\label{enum:StructuralObjective}
\end{enumerate}

In the following section I will describe more in detail the structure of the \textit{JointRobotTP} class.



% definire le parti della classe fare uno schema dei componenti
The unified robot class implementation main \textit{methods} and \textit{data structure} are:
%\begin{itemize}
%	\item \textbf{Initialization}
%					-- > initialize
%					-- > prxtask call back
%	\item \textbf{Data structure}
%					-- > init configuration
%					-- > prox task points
%					-- > TP_ task map
%	\item \textbf{Task update methods}
%					-- > description of the logic
%	\item \textbf{Action Server}
%                   -- > execute
%%	\item \textbf{Movement methods}
%					-- > Reach initial configuration
%					-- > Run Cartesian Reaching Loop
%\end{itemize}



\subsection{Class setup}
The \textit{Initialize} method is called once to instantiate all the members of the class object, data structure, publishers, subscriber and the action server.
\begin{figure}[H]
	\centering
	\includegraphics[scale=0.5]{initialize_diagram.png}
	\caption{activity diagram of the \codetxt{olive}{initialize} method}
	\label{fig:initialize_diagram}
\end{figure}
\noindent In the first step, the the robots nodes \textit{kuka\_robot\_node} and \textit{ur10\_robot\_node} are created and initialized with their class-specific method, used for interface configuration and internal state initialization.\\
\noindent Next the communication interfaces are created, the \textit{proximity task} subscriber used to recive data from the \textit{Environment handling} stak \ref{fig:simulation_nodes_diag}. And the \textit{Action server}, used to send \textit{goals} in \textit{position} and \textit{orientation} to the unified robot. \\
\noindent For the last step, the method handles the initialization of a series of variables and data structures used later in the node.
\begin{itemize}
	\item \textbf{Parameters}: all the gains, threshold and other run-time variable are created as \textit{ros2 parameters}.
	\item \textbf{Initial joint configuration}: stored as a \codetxt{darkgray}{std::map} the method populate a list of vectors containing potential initial configuration.
	\item \textbf{Task update function}: the method stores the pointers to the update of the active \textit{tasks}.
\end{itemize}
\noindent
This structure was chosen to have a more manageable code and to accomplish the second objective \ref{enum:StructuralObjective} I set for the architecture.

\subsection{Task state update}
Just to explain the logic in my implementation, for each task defined for a particular objective in the control algorithm I developed three different methods used to cyclically update the information contained in the \textit{task data structure}. \\
Than the pointers to these methods are inserted in a vector, and this vector is used to call all the function in a loop. This is functional since I can remove a task from the update cycle just by commenting three rows, and I have a clear idea of which task I am updating in a very concise way.

\subsection{Task data structure}
For the control algorithm I had to define three different matrices for each task I created. I decided to implement a \textit{stuct} to store the matrices and put everything in a map(\codetxt{darkgray}{map<string, tp\_task>} \codetxt{cyan}{TP\_task\_map\_}\codetxt{darkgray}{;}).
\\
\noindent
Saind \textit{struct} is defined as follows:
\\[6pt]
\noindent\makebox[\textwidth]{
	\begin{tabular}{l}
		\codetxt{darkgray}{Eigen::MatrixXd} \codetxt{cyan}{RefRate}\codetxt{darkgray}{;}\\[4pt]
		\codetxt{darkgray}{Eigen::MatrixXd} \codetxt{cyan}{ActMatrix}\codetxt{darkgray}{;}\\[4pt]
		\codetxt{darkgray}{Eigen::MatrixXd} \codetxt{cyan}{TskJacobian}\codetxt{darkgray}{;}
	\end{tabular}
}
\noindent This definition uses dynamically sized matrices that are useful in the case of the \textbf{task priority} control since the dimension of each matrix can change from task to task, maintaining a certain relation within each task:
\begin{itemize}
	\item $\textit{RefRate}\in \mathbb{R}^{(m\times1)}$
	\item $\textit{ActMatrix} \in \mathbb{R}^{(m\times m)}$
	\item $\textit{TskJacobian} \in \mathbb{R}^{(m\times n)}$
\end{itemize}
\noindent The use of \codetxt{darkgray}{std::map} also allows to reference to each task trough a \textit{string} witch is a very flexible and efficient approach(search complexity $O(\log n)$, with n the element number in the map).


\section{Unified Robot Architecture: Action server}
The action server in the \textit{ros2} framework is composed of various methods, used for the \textit{acceptance}, \textit{rejection}, and \textit{abortion} of goals sent to the server. The main body of the server instead is located in two methods called: \codetxt{olive}{execute}, and \codetxt{olive}{RunCartesianReachingLoop}.

\subsection{execute method}
This method is executed in a separate thread, created after the goal has been accepted. Firstly from the \textit{goal\_handle} message the \textit{initial joint confiuration code} is retrived and a simple control loop is than used to control the joint:
\begin{figure}[H]
	\centering
	\includegraphics[scale=0.4]{JointControlLoopDiagram.png}
	\caption{Joint reaching control loop}
	\label{fig:JointControlLoop_flowchart}
\end{figure}
\noindent
the loop drives each joint towards the desired configuration. The integration of this part in my architecture was done improve repeatability, to have the possibility to impose a common starting position in different experiments.\\
\noindent The second procedure is the projection of the \textit{requested goal} in the correct \textit{reference frame}, and after that said goal is broadcasted in Rviz using a publisher.
\noindent The core functionalities of the \codetxt{olive}{execute}\codetxt{darkgray}{();} method are here represented in a flow-chart to highlight better the loop inner workings:

\begin{figure}[H]
	\centering
	\includegraphics[scale=0.5]{Action_server_flowChart.png}
	\caption{Action server flow chart}
	\label{fig:ActionServer_flowchart}
\end{figure}



\section{Unified Robot Architecture: Control alorithm}
% TP computation class description
The structure of the control algorithm was adapted to the \textit{task state} data structure, I wanted to have in the control loop a very clear way to identify which tasks I was activating. And to \textit{passively} impose the priority without needing to use a flag.
\\
With this in mind I created a class called: \codetxt{darkgray}{TPComputation}, of witch I could instantiate an object for each iteration of the reaching loop, the class data would be the \textit{null space} of the \textit{task jacobian} computed at the previous step, $Q_{k-1}$, and the resulting \textit{joint velocity} vector, $\rho_{k-1}$. These data would be extracted from the class as needed to send the command to the robot.\\
\noindent
To initialize the class object the method \\ \codetxt{olive}{init\_TPComputation} \codetxt{darkgray}{([Args...]);} takes as arguments the number of \textit{Dof} of the structure, used to initialize the dimension of the internal matrices and some parameter that are used in the pseudo-inversion of the \textit{augmented jacobian}.
Since I wanted to be sure that no information remained after any cycle, I also added the method \\ \codetxt{olive}{kill\_TPComputation} \codetxt{darkgray}{();} that \textit{clears} the internal variables after use.
\\
\noindent
The step computation of the \textit{task priority inverse kinematics} is performed using the \codetxt{olive}{computeTP\_step} \codetxt{darkgray}{([Args...]);} method. The arguments taken by the function are the matrices of the desired task, stored in \codetxt{darkgray}{map<string, tp\_task>} \codetxt{cyan}{TP\_task\_map\_}\codetxt{darkgray}{;}.
\\ Internally the computation is than performed using also:\\ \codetxt{olive}{REG\_Pinv\_operator} \codetxt{darkgray}{([Args...]);} and \codetxt{olive}{REG\_Pinv} \codetxt{darkgray}{([Args...]);} but I will better discuss the technique used in chap. \ref{chap:fourth}. Lastly after all the desired steps have been computed, the resulting \textit{joint velocity vector} is extracted using: \codetxt{olive}{getTP\_ydot} \codetxt{darkgray}{();}.

\subsection{Reaching loop process flow}
To better show the intended and actual use of the \codetxt{darkgray}{TPComputation} class to control the robot, I added the flow-chart for the \codetxt{olive}{RunCartesianReachingLoop}\codetxt{darkgray}{([Args..]);} method. 

\begin{figure}[H]
	\centering
	\includegraphics[scale=0.4]{Reaching_loop_flowChart.png}
	\caption{Reaching loop flow chart}
	\label{fig:ReachingLoop_flowchart}
\end{figure}

\noindent
To briefly describe the loop, after the instantiation of the \codetxt{darkgray}{TPComputation} object the loop starts.  In the \textit{Update step}, all the function that are relative to a \textit{task} are called, starting with the \textit{Reference Rate}, \textit{Activation function} and lastly \textit{Task Jacobian}.
The second part, the \textit{Stop condition} is checked. If the control is positive the loop is immediately stopped, and the result is sent to the action server client.
if the opposite is true, the \textit{tp\_controller} is initialized, with the \textit{Dof} of the system and the variables for the pseudo-inverse computation. Than each "\textit{step}" of the algorithm is performed, finishing with a "\textit{null}" task, composed of two Identity matrix for \textit{Activation function} and \textit{Task Jacobian}, and with a zero vector for \textit{Reference Rate}.
The loop is repeated until the stop condition is met or stopped after a time limit, in this case the goal is considered not reached.



