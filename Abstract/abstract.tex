%%%%%%%%%%%%%%%%%%%%%%%%%%%%%%%%%%%%%%%%%%%%%%%%%%%%%%%%%%%%%%%%%%%%%%%%%%%%%%%%
%2345678901234567890123456789012345678901234567890123456789012345678901234567890
%        1         2         3         4         5         6         7         8
% THESIS ABSTRACT

% Use the following style if the abstract is long:
%\begin{abstractslong}
%\end{abstractslong}

\begin{abstracts}

In industrial application the use of robotic systems has constantly increase from the sixties to today. But even with all the the technological improvements of the later years some tasks are still to complex or difficult to automatize completely.
For this reason the research moved towards the "collaboration" between robot and machines. 
Two of the main challenges of this approach are: safety and ease-of-collaboration, meaning the robot have to be able to recognize the operator and enact safety procedure to avoid injury but also the human agent need to interact with the robot in the most natural way possible to increase the overall performance and the effectiveness of the couple human-machine.
On this topic the MACLAB laboratory developed an initial demo for a collaborative workcell, within the SESTOSENSO project, to install the canopy of a car.
My contribution to the project, and the objective of this thesis, was to explore the capabilities of the robotic system as one singole entity. Since the real system is composed of two industrial robot mounted in series, and since the dynamics controller was  a proprietary one, the MACLAB decidet to opt for a simpler cinematic controller.
My work was conducted in a simulation enivironment developed for the control of the real robot, and this allowed me to explore a more complex control scheme.
I moved to a Task Priority approach since I wanted to effectively use all the DOF of the robot, to achive different objectives.
The main task was to evaluate the behaviour of the robot in a task of obstacle avoidance, I simulated the proximity sensor used in the real system, and evaluated the behaviour with a series of experiment.

\end{abstracts}
