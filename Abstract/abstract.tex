%%%%%%%%%%%%%%%%%%%%%%%%%%%%%%%%%%%%%%%%%%%%%%%%%%%%%%%%%%%%%%%%%%%%%%%%%%%%%%%%
%2345678901234567890123456789012345678901234567890123456789012345678901234567890
%        1         2         3         4         5         6         7         8
% THESIS ABSTRACT

% Use the following style if the abstract is long:
%\begin{abstractslong}
%\end{abstractslong}

\begin{abstracts}

Since the 1960s, the use of robotic systems in industrial applications has continuously increased. However, even with this incredible force driving innovation, some tasks have proven to be too complex or not cost-effective to be performed by a robot.
With the advent of Industry 4.0, the proposed solution to these problems was \textbf{Human-Robot Collaboration} — building work-cells capable of integrating a human agent performing a set of tasks that can be coordinated with a robotic agent to achieve a common objective.
This approach opened up a completely new set of challenges, the first of which are safety and perception.
The robotic agent needs a way to perceive the human in the workcell and must be able to react to unpredictable movements to avoid collisions.
During my thesis, I worked within the \textbf{SESTOSENSO project}, specifically in Use Case 1. Their robotic system, composed of two 6-DoF industrial articulated robots mounted in series, is equipped with a set of proximity and tactile sensors.
My work focused on creating a unified architecture for the two robots, exploring the capabilities of a 12-DoF robot, and proposing possible directions to improve the system’s functionalities. Moreover, this work also aimed to identify potential problems and weaknesses.
I achieved these objectives through a series of simulated experiments, using a task-priority approach for system control, as I was interested in exploiting the high redundancy of the robot to perform multiple tasks simultaneously. I than analyzed the result to evaluate the effect of each task on the behavior of the robot.


\end{abstracts}
