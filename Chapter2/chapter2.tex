%%%%%%%%%%%%%%%%%%%%%%%%%%%%%%%%%%%%%%%%%%%%%%%%%%%%%%%%%%%%%%%%%%%%%%%%%%%%%%%%
%2345678901234567890123456789012345678901234567890123456789012345678901234567890
%        1         2         3         4         5         6         7         8
% THESIS CHAPTER

\chapter{State of the art}
\label{chap:second}
\graphicspath{{Chapter2/Figures/PNG/}{Chapter2/Figures/JPG/}{Chapter2/Figures/JPEG/}{Chapter2/Figures/}}

In this chapter I will explore the literature present on the topic of: \textit{Obstacle avoidance} since is one of the focus of my thesis, and a key aspect of Human-Robot interaction. And also on the topic of \textit{High redundancy robot}, I want to find if someone else has done something with a structure similar to the one of the one of the \textbf{SestoSenso project}. 

\section{Obstacle Avoidance}
Obstacle avoidance is one of the most impact full challenges to overcome when it comes to \textit{Human-Robot interaction}. It requires the robot to perceive the environment in some way, and use the information to modify its behavior to accomplish a goal.
 
There are two main field of \textit{obstacle avoidance} that we can recognize that are based on the type of robots we are dealing with:
\begin{itemize}
	\item obstacle avoidance for mobile robots.
	\item obstacle avoidance for robotic manipulator.
\end{itemize} 
I explored some of the methods used for the second case, since the first is outside the scope of this thesis.
\subsection{Camera based methods}
As reported in \cite{VISIONBASED_OBREC}, we can divide the camera based methods in two main categories:
\begin{itemize}
	\item Monocular vision: use a single camera mounted on top of the robot.
	\item Stereo vision: use two synchronized cameras.
\end{itemize}
% MONOCULAR CAMERA
The basic approach of the \textit{visual servoing} with monocular camera can be represented in the following scheme:
\begin{figure}[H]
	\centering
	\includegraphics[scale=5.0]{Direct-visual-servoing-scheme.jpeg}
	\caption{Direct visual servoing scheme}
	\label{fig:DirectVisualScheme}
\end{figure}
In the work of \cite{OBJREC_ARM_MONOCAM} we can see how a monocular system is used with a \textit{deep Region based Convolutional Neural Network} to recognize objects in the field of view of the robot and decide if said object is a target or not.
This first step is than followed by a \textit{kNN} and the \textit{Fuzzy interference system} to localize 2D position of the targets, the last coordinate is found by only shifting the end-effector a few millimeters towards the x-axis. 
Another work worth mentioning, that is also use to do \textit{obstacle avoidance} is the one done by \cite{STRAWBERRY_RECOGNITION}.



% STEREO VISION

% NN IMAGE RECOGNITION

\section{High DoF architecture}