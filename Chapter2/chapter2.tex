%%%%%%%%%%%%%%%%%%%%%%%%%%%%%%%%%%%%%%%%%%%%%%%%%%%%%%%%%%%%%%%%%%%%%%%%%%%%%%%%
%2345678901234567890123456789012345678901234567890123456789012345678901234567890
%        1         2         3         4         5         6         7         8
% THESIS CHAPTER
\chapter{State of the art}
\label{chap:second}
\graphicspath{{Chapter2/Figures/PNG/}{Chapter2/Figures/JPG/}{Chapter2/Figures/JPEG/}{Chapter2/Figures/}}

In this chapter I will explore the literature on the topics of: \textit{ambient perception} to expore different sensing approaches to control manipulators in a dynamic environment. I will than focus my attention on the uses of this sensory information in the task of \textit{obstacle avoidance}
\noindent 
During my state of the art research it was apparent a lack of literature on the specific architecture developed in the \textit{SestoSenso project}. For this reason the last part of this chapter has focused more broadly on \textit{high dof architecure}, to explore how long open kinematic chain system are treated, and in which areas are used.

\section{Environment perception and awareness}
Environment perception is one of the biggest difference when we move from the classical use of robotic systems in industry, to a more modern framework geared towards HRI.
In this section I reported two of the main method for extracting ambient morphology information from various types of sensors, and explained their strength and weaknesses.

\subsection{Image recognition based methods}
As reported in \cite{VISIONBASED_OBREC}, we can divide vision based methods in two main categories:
\begin{itemize}
	\item Monocular vision: use a single camera mounted on top of the robot.
	\item Stereo vision: use two synchronized cameras.
\end{itemize}
% MONOCULAR CAMERA
The basic approach of the \textit{visual servoing} with monocular camera can be represented in the following schema:
\begin{figure}[H]
	\centering
	\includegraphics[scale=5.0]{Direct-visual-servoing-scheme.jpeg}
	\caption{Direct visual servoing scheme}
	\label{fig:DirectVisualSchema}
\end{figure}
In the work of \cite{OBJREC_ARM_MONOCAM} we can see how a monocular system is used with a \textit{deep Region based Convolutional Neural Network} to recognize objects in the field of view of the robot and decide if said object is a target or not.
This first step is than followed by a \textit{kNN} and the \textit{Fuzzy interference system} to localize 2D position of the targets, the last coordinate is found by only shifting the end-effector a few millimeters towards the x-axis.

% STEREO VISION
Following and improving the capabilities of a singular camera system there is the use of: stereo vision.
Stereo vision works by combining the information of the two cameras, that are placed in a known position to extract information of the third dimension of objects in the images.
In the work of \cite{huh2008stereo} we can see how a stereo vision based system is used to perform obstacle recognition on a autonomous driving vehicle.
\begin{figure}[H]
	\centering
	\includegraphics[scale=0.4]{stereoVision_schema.png}
	\caption{Vision coordinate sistem}
	\label{fig:StereoVisionSchema}
\end{figure}
The image based methods in general appear to have some key characteristics that make them impractical for effective \textit{obstacle perception} in an environment such as the one of the \textbf{SestoSenso project}.

With the advent of AI in recent years the use and potential of \textit{image recognition} and with it all the vision based systems has greately increased.
But they still have a series of weaknesses that have a large impact when it comes to develop viable systems for industrial application.
In the work of \cite{s25206402} a control system using a 3d stereo camera and the YOLO AI algorithm for image recognition, those limitation are evident:
\begin{itemize}
	\item \textbf{Hardware and software}: The cost of the system parts is not paltry, form the camera to the AI software.
	\item \textbf{Integration and configuration}: The camera apparatus has an implementation that is task-specific, which means that every change in the environment requires a complete re-configuration of the system.
	\item \textbf{Computational cost}: The image analysis is performed on a separated computer to manage the burden of the computation.
	\item \textbf{Field of view}: The camera visualize only the work area, which is not adequate for a HRI situation.
	\item \textbf{Environment interference}: The use of 2D and 3D image information, require to have a strict control on the occlusion and disturbances in the environment, from lighting to airborne dust. This level of control is not possible in a industrial context. 
	\item \textbf{Privacy}: One problem not addressed by the paper is the privacy of people working around or with the robot, that is not mantained with the use of a camera.
\end{itemize}

\subsection{Point-cloud discretization based methods}
In the work of \cite{zauner2025workspace} three different type of spatial perception sensor are evaluated to create point cloud of a robot's workspace. To perform safe navigation and avoid collisions.
the sensor used are:
\begin{itemize}
	\item \textbf{Time Of Flight}: \textit{Kinect V2} and \textit{Omron OS32C Lidar}
	\item \textbf{Active Stereoscopy}: \textit{Intel RealSense D435}
\end{itemize}
The two time of flight sensor work with a infrared light and a laser respectively and the measure the distance from an object by timing the time delta at the reception of the light impulse.
The Intel sensor instead is based on the \textit{stereo vision} principle, but it uses simpler cameras aided by a infrared projector that imposes a grid of points onto the surfaces.
\\
\noindent
The sensors are mounted on the \textit{end effector} of the manipulator and panned over the workspace to record a sample of the environment, the resulting pointcloud is than processed to reduce the number of points and to extract feature of the environment.
\\
\noindent
In the paper the extracted feature are used to simplify the 3D representation of the obstacle, and to perform collision-checks they confronted a series of different algorithms. In the case of my thesis I stopped after the filtering to reduce the number of points, than the point cloud is directly used to represent the robot and obstacle in the simulation.
\\
As stated in \cite{hussmann2008performance} the main drawback of \textit{ToF} sensors is the lower resolution capabilities in comparison to \textit{stereo vision} techniques. The paper highlight that even with this performance deficit the \textit{ToF} were viable to be used in automotive application even for safety tasks.


\section{Obstacle avoidance in HRC}
For \textit{Human--Robot Collaboration} applications, the robot must operate under a \textit{multi-objective} control strategy, where the system handles a \textit{goal-driven task} defining the role of the robot, and one or more other task that go from safety to optimization tasks.  
Within collaborative scenarios, the safety layer must be treated with \textbf{higher priority}, temporarily overriding the main objective whenever a hazardous situation is detected, to guarantee human and system protection in real time.  
In this thesis, the prioritized secondary objective ensuring safe collaboration is \textit{obstacle avoidance}, which monitors the robot surroundings and generates motion corrections when the robot is to close to the obstacle.

In the case of a manipulator arm we have to also include the $z$ axis, since we are operating in 3d space.
Looking at the work of \cite{zhang2003obstacle} we can see how we can compute a safe trajectory for a \textit{SIR-1} robot manipulator using cubic polynomials for a path with intermediate points.
In this paper is interesting the introduction of the concept:\textit{link} collision avoidance. By controlling the \textit{link} closest point to the obstacle and using the analytical formulation of the \textit{Inverse Kinematics} and the \textit{obstacle shape} it is possible to define \textit{joint variables} constraints to ensure a collision free navigation.\\
\noindent
In the work of: \cite{maciejewski1985obstacle} instead the proposed approach considers the closest point of the hole robot to the obstacle, than it apply a velocity vector to said point that is directly opposite to the distance vector $(P_{ob}-P_{rb})$.  
\begin{figure}[H]
	\centering
	\includegraphics[scale=0.35]{obav_scheme.png}
	\caption{Obstacle avoidance schema}
	\label{fig:OBAVSchema}
\end{figure}
To ensure that the \textit{obstacle avoidance} desired velocity does not impact the tracking of the \textit{end effector velocity} the joint space velocity are searched in the null-space of the solution to the first problem. This approach allows to leverage the redundancy of the manipulator.
The proposed algorithm is than applied to a planar robot with parallel \textit{revolute} joints, as shown in the image. And also to a 3D redundant manipulator operating through the window of an automobile door.

\subsection{Redundancy control}
From the discussion of the previous section it is clear that to correctly perform obstacle avoidance the control architecture has to deal with multiple objectives. This objective need to be \textit{task oriented} to allow for portability between different system configuration.
The classic framework for this type of control was developed by \cite{slotine1991general} and extended by \cite{simetti2016novel} to include the activation and deactivation of task without discontinuities.
The general idea is to have a \textit{hierarchy of tasks}, defined to be \textit{objective specific} and not connected to the particular structure of the robot.
Given a generic objective function defined in the task space:
\begin{equation}
	\vec{\dot{x}}_i = \vec{J}_i(q) \cdot \vec{\dot{q}}
\end{equation}
\begin{itemize}
	\item $\vec{\dot{q}} \in \mathbb{R}^{(n\times1)}$: joint displacement vector.
	\item $\vec{\dot{x}}_i \in \mathbb{R}^{(m_i\times 1)}$: task velocity vector, or \textit{reference rate}.
	\item $\vec{J}_i(q) \in \mathbb{R}^{(m_i\times n)}$: task jacobian matrix.
\end{itemize}
Given that the solution of the highest priority task is:\\ $\vec{\dot{q}}_1 = \vec{J}^{\#}_1\vec{\dot{x}}_1 + (\vec{I} - \vec{J}^{\#}_1\vec{J}_{1})\vec{\dot{z}}, \forall \vec{\dot{z}}$ the second part of the solution is the projector on the \textit{null space} of $\vec{J}_{1}$, the solution to the lower priority task are searched in that space.
Yielding the general solution:
\begin{equation}
	\vec{\dot{q}}_i = \vec{\dot{q}}_{i-1} + \vec{J}_{i}(\vec{I} - \vec{J}^{\#}_i\vec{J}_{i})(\vec{\dot{x}}_i - \vec{J}_{i}\vec{\dot{q}}_{i-1})
	\label{equat:TaskPrioritiClassic_generalSolution}
\end{equation}
In the paper is demonstrated that the solution of a lower priority task does not modify the higher one, but it is \textit{attempted} in the null space.



\section{High DoF architecture}
In this section I want to explore some of the relevant high-dof architecture found in the literature.
\subsection{Dual-arm systems}
In recent years there has been a trend to use these dual-arm systems for HRC(Human Robot Collaboration), but also for replacing human workers without the need to redesign the work cell.
\begin{figure}[H]
	\centering
	\includegraphics[scale=1.5]{SDA10_500_dualArmManipulator.jpg}
	\caption{Dual-arm industrial robot example, SDA10}
	\label{fig:DualArmRobot}
\end{figure} 
As is stated in the survey of \cite{DUAL_ARM_SURVEY} the strengths of the dual arm architecture are:
\begin{itemize}
	\item \textit{Similarty to operator}: useful both in the case of HRC and to substitute the human worker with minimal effort.
	\item \textit{Flexibility and stiffness}: Combining the stiffness of closed chain manipulation, with the flexibility of a serial link.
	\item \textit{Manipulability}: High number of DoFs allows for complex motion tasks.
	\item \textit{Cognitive motivation}: The similar charcteristics of the kinematic chain is belived to be helpful in HRC context.
\end{itemize} 
In most cases for these architectures the \textit{obstacle avoidance} is computed for the \textit{navigation} if the robot has a movable base.
Moreover the interaction with the environment is performed with the use of \textit{visual servoing}, witch was firstly discussed by \cite{hutchinson2002tutorial}, position based and \textit{hybrid} methods, combining visual and position information.
\subsection{Snake-like robot}
A completely different class of robot is represented by the "\textit{snake like}" robot.
As shown in the work of \cite{hirose2009snake} and \cite{crespi2005amphibot} these types of robot are biologically inspired, and they can produce a forward motion from an undulatory one. Reproducing the movement patterns of snakes.
\begin{figure}[H]
	\centering
	\includegraphics[scale=1.5]{anphiBot_snakelike.jpg}
	\caption{snake like robot from \cite{crespi2005amphibot}}
	\label{fig:snakeLikeRobot}
\end{figure} 
The potential of these robot that are currently being explored are for navigation in tight spaces, to be applied to endoscopes for examples. In addition the interest lies in the flexibility of a "\textit{snake like}" body, since it could be used to move, climb and grasp if needed.
\\
\noindent
\subsection{Planar robot}
For more industrial application, I reviewed the work of \cite{highDof_planar_robot_OBAV} and \cite{highDof_planar_robot_OBAV_2}. These work take into consideration high-dof planar manipulator, and they also propose two approaches to do \textit{Obstacle avoidance} with their respective architecture.
In the case of  \cite{highDof_planar_robot_OBAV} the paper uses the \textit{ANAT} robot, presented in the figure below.
\begin{figure}[H]
	\centering
	\includegraphics[scale=0.45]{ANAT-robot-arm.jpeg}
	\caption{ANAT robot arm}
	\label{fig:AnatRobot}
\end{figure}
\noindent
This is a 7Dof robot, comprised of 1 \textit{prismatic} joint to control the $z$ cordinate, followed by 3 parallel \textit{revolute} joints and a 3 Dof \textit{wrist}. \\
\noindent
The proposed control algorithm is based on the work of \cite{zlajpah1997control} for the computation of the generalized inverse of the jacobian matrix. In addition, the obstacles are modeled as hyper planes to reduce computational costs and the control law is applied to the joints in order.
In this paper the proposed method is applied at the \textit{Dynamic} level, the objective function for the \textit{Obstacle avoidance} is computed as follows:
\begin{equation}
	V_1(q)=
	\sum_{i=1}^{m}\sum_{j=2}^{n}
	\frac{\alpha_{ij}}{
		-(\frac{x_j - x_{ci}}{r_i + r_{si}})^2
		-(\frac{y_j - y_{ci}}{r_i + r_{si}})^2
		-(\frac{z_j}{h_i + h_{si}})^2
		+1 \;
		\vphantom{\Bigl( \frac{A}{B} \Bigr)}
	}
	\label{equat:Obstacle Avoidance ANAT robot}
\end{equation}
where:
\begin{itemize}
	\item \textit{m},\textit{n}: respectively the number of obstacles, and the number of points placed on the robot.
	\item $\alpha_{ij}$ : weight of the constraint for joint \textit{i} from obstacle \textit{j}
	\item $(x_j,y_j,z_j)$ : coordinates of joint \textit{j} in the \textit{base frame}
	\item $(x_{ci},y_{ci},r_i,h_i)$ : coordinates of cylinder \textit{i} in the \textit{base frame}
	\item $(r_{si},h_{si})$ : safety distances in \textit{radius} and  \textit{height} from cylinder \textit{j}
\end{itemize}
The approach generates a \textit{repulsive force} that becomes stronger as the robot approaches an obstacle.
While \textit{potential-field techniques} are a standard choice for \textit{dynamic obstacle-avoidance control}, I could not adopt them in this work because the robots' \textit{dynamic controllers} were locked behind the manufacturer's proprietary software, preventing access to the required control layer.\\


\subsection{Macro Micro configuration}
The last interesting configuration I want to talk about is referred in the literature as \textit{Macro-Micro Robot}. Firstly proposed by \cite{sharon1984enhancement}, the objectives of the proposed architecture were to resolve the opposing problems of \textit{speed} and \textit{tracking precision} and also to correct the errors in \textit{end point measurement}, given by bending in the links and errors in the measurement errors in the encoders.\\

The uses and capabilities of this configuration are presented in the work of \cite{MacroMicro_sampling}, following is a photo of the robot they used.
\begin{figure}[H]
	\centering
	\includegraphics[scale=0.45]{macro_micro_example.png}
	\caption{Macro-Micro robot}
	\label{fig:MacroMicroRobot}
\end{figure}
The robot is composed of a $6$ Dof \textit{Macro} manipulator and a $3$ Dof \textit{Micro}. The robot is equipped to perform polishing tasks on complex surfaces. \\
\noindent
The paper focus is to prove the effectivenss of a \textit{sampling based} motion assignment(MA) strategy with multi performance optimization. Since the robot has a total of $9$ degree of freedom there is space for optimization in the robot movement.
%The cost function to be minimized is expressed here:
%\begin{equation}
%	f_c(q) = \sum_{i=1}^{l}(w_i \cdot RPI_{c,i}(q))
%	\label{equat:MacroMicro_minimz_fnc} 
%\end{equation}
The configuration optimization function is to be minimized for each sampling point of the chosen trajectory, and for each point the performance index and constraint ($RPI_{c,i}(q)$) must be computed. 
Classic \textit{gradient based} methods can easly stop at local minima since the function is not convex, the proposed MA aims to optimize the movement of macro and micro manipulator, avoiding the costly and error-prone computation.
The system on witch I worked on this thesis is of the same general structure, but the two robot are considered as a whole. Also in my work I am not computing any offline trajectory as in the case of this paper.\\
\noindent
Another application of the \textit{Macro-Micro} configuration is in the field of medical robotics, in the work of \cite{MacroMicro_surical}. In this paper the proposed architecture is composed of a \textit{KUKA LBR IIWA} robotic arm with $7$ Dof, and a \textit{Micro-IGES} surgical robotic instrument with $7$ Dof($2$ Dof are composed of the jaws of the instrument).
\begin{figure}[H]
	\centering
	\includegraphics[scale=0.45]{MacroMicro_surgical.png}
	\caption{Macro-Micro surgical robot}
	\label{fig:MacroMicroSurgicalRobot}
\end{figure}
\noindent
In this paper the objective was to demonstrate that the overall performance of the system can be improved by defining preoperatively the best initial configuration of the surgical instrument in terms of \textit{roll}, \textit{pitch} and \textit{yaw} with respect to the macro serial-link manipulator to achive maximum accuracy in performing specified tasks.
The paper highlights how the macro micro manipulator configuration allows for completion of multiple-objective tasks, such as:
\begin{itemize}
	\item \textit{Guarantee Remote Center of Motion}: The RCM(which for surgical application is usually the incision site) has to remain stationary.
	\item \textit{Desired path tracking}
	\item \textit{Assembly errors compensation}
\end{itemize} 
The method used in this paper starts with a \textit{Genetic algorithm} used to generate possible configuration, that are than evaluated through \textit{Hierarchical Quadratic Programming}.
The solution of the procedure finds the best intial configuration based on a fitness function and resiliance to errors.


