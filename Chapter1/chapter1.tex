%%%%%%%%%%%%%%%%%%%%%%%%%%%%%%%%%%%%%%%%%%%%%%%%%%%%%%%%%%%%%%%%%%%%%%%%%%%%%%%%
%2345678901234567890123456789012345678901234567890123456789012345678901234567890
%        1         2         3         4         5         6         7         8
% THESIS CHAPTER

\chapter{Introduction}
\label{chap:first}

Robotic systems from their first introduction in the manufacturing field we relegated to work separated from the human workers. This was because the main use for robots was to perform, highly repetitive tasks, very fast or to work in dangerous environment. This removed the need to have interaction between human and robots.
With the advent of industry "3.0" and "4.0" the focus shifted from that to have the robots collaborate with humans to increase efficiency, and to remove some burden from the human worker, especially for physically demanding tasks.

The concept of Human Robot Interaction (HRI) appears in the literature and can be divided in two broad categories, each with their respective challenges.
\begin{itemize}
	\item \textbf{Physical Interaction}: interaction that require or could have some form of contacts with the robotic system.
	\item \textbf{Social Interaction}: interaction that aims to exchange information, or perform conversation of some kind.
\end{itemize}
In the context of this thesis, and more broadly in industrial applications, the focus is primarily on physical interaction. Collaborative robots operating alongside human workers must function in dynamic environments, where the human agent does not follow predefined trajectories. \\


My thesis was defined alongside the \textbf{SestoSenso Project} which proposes a framework for Human-Robot Collaboration in which controlled physical contact is not only possible but expected. Within this framework, the robot and the human operator jointly manipulate or work on the same object, requiring the robotic system to adapt continuously to the human’s actions.

The main objective of the project was to enable sensor-based control and human-robot collaboration with an industrial high-payload manipulator, that is normally used only in closed-off environment.

To support this type of architecture, the robotic platform is equipped with proximity sensors that allow it to perceive changes in its surroundings and react autonomously and in real time. In addition, several robot links are covered with a sensorized tactile skin, enabling the system to detect and interpret physical contact with the environment or with the human collaborator.

A key strength and novelty of the SestoSenso robotic setup lies in its multi-stage structure. The complete system features 12 degrees of freedom, created by combining two manipulators: a high-payload industrial arm from KUKA as the first stage, and a lightweight, highly compliant arm from Universal Robots as the second stage. This configuration allows the robot to leverage the strengths of both manipulators, power from the KUKA arm, and flexibility and precision from the UR arm, making it well suited for collaborative tasks.




\section{Research problem and objective}
% quali sono i problemi che deve affrontare questo sistema
Since during the \textbf{SestoSenso Project} the two robots were controlled separately, due to technical complication in the control of the real robots, in this work I developed a unified control architecture with the aim to explore the capabilities of the complete kinematic chain controlled with a reactive control schema. Specifically with \textit{reaching} and \textit{obstacle avoidance} tasks.
All the activities were carried out at MACLAB, the Mechatronics and Automatic Control Laboratory at Università degli Studi di Genova.

\section{Thesis structure}
After the brief introduction in chap.\ref{chap:first} of the objective of this thesis, in chap. \ref{chap:second} I will provide the state of the art for: technologies used for ambient sensing, their strength and weaknesses related to this specific application. 
\noindent Approaches of \textit{obstacle avoidance} for manipulators, and redundancy control. Finally a general discussion on various types of High-Dof structures. 
The chap. \ref{chap:third} will be a discussion on the mathematical foundation of algorithms and methods I used in my work.
In chap.\ref{chap:fourth} instead the focus will be on the architecture implementation, describing the functionalities I included and how the system works.
Lastly \ref{chap:fifth} will be the presentation of the simulation environment and the experiments I carried out. The conclusion are in chap.\ref{chap:conclusions}.
 